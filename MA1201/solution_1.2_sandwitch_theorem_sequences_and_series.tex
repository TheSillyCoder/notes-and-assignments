\documentclass[10pt]{article}

\usepackage[T1]{fontenc}
\usepackage{geometry}
\usepackage{graphicx}
\usepackage{amssymb}
\usepackage{amsmath}
\usepackage{enumitem}
\usepackage[english]{babel}
\usepackage{amsthm}

\newtheorem*{lemma}{Lemma}
\title{cancellation properties}
\author{Debayan Sarkar}
\date{\today}

\geometry{a4paper, margin=0.75in}
\setlength{\parindent}{0pt}
\newcommand{\nn}{\mathbb{N}}
\newcommand{\inflim}{\lim_{n\to\infty}}
\newcommand{\sol}{\textbf{Ans : }}

\begin{document}
        \par\textbf{IISER Kolkata} \hfill \textbf{MA1201}
        \vspace{3pt}
        \hrule
        \vspace{3pt}
        \begin{center}
                \LARGE{\textbf{Solutions to Assignment 1.2}}
        \end{center}
        \vspace{3pt}
        \hrule
        \vspace{3pt}
        Debayan Sarkar, \texttt{22MS002}\hfill\today
        \vspace{20pt}\\
        \textbf{Problem 1.}
        \begin{enumerate}[label = {(\alph*)}]
        \item Let $x_n$ be a convergent sequence of reals and there exists $m \in N$ such that $x_n \geq 0 \,\,\, \forall n \geq m$. Prove that $$ \lim_{n\to\infty}x_n \geq 0$$
        \sol Let's assume to the contrary that, $$x:=\lim_{n\to\infty} x_n < 0$$ Then, for $\epsilon = -\frac{x}{2} > 0 \,\,\, \exists k \in \mathbb{N}$ such that $K > max\{K(\epsilon), m\}$. Now, $\forall n \geq k$, we have $x_n \geq 0$ and $$x - \epsilon < x_n < x + \epsilon = x - \frac{x}{2} = \frac{x}{2} < 0 \Rightarrow x_n < 0 $$ which is a contradiction. Hence, $x \geq 0$ or, $$\lim_{n\to\infty} x_n \geq 0$$ \qed \\ We are going to use this as lemma 1.a in the following proofs.
        \item Let $x_n$ and $y_n$ be two convergent sequences of reals and $\exists m \in \mathbb{N}$ such that $x_n \geq y_n \,\,\, \forall n \geq m$. Prove that $$\lim_{n\to\infty} x_n \geq \lim_{n\to\infty} y_n$$
        \sol Let's define 
        \begin{align*}
            &x : = \inflim x_n\\
            &y : = \inflim y_n\\
            &z_n := x_n - y_n\\
        \end{align*}
        Then we have,
        \begin{align*}
            z := \inflim z_n = \inflim x_n - \inflim y_n  = x - y 
        \end{align*}
        Also, $z_n \geq 0 \,\,\,\forall \,\,\,n \geq m$. Using lemma 1.a on $z_n$ we know that $z \geq 0 \Rightarrow x - y \geq 0 \Rightarrow x \geq y$ or, $$\inflim x_n \geq \inflim y_n$$ \qed \\ We are going to use this as lemma 1.b for the following proofs.
        \item If $x_n$ is a convergent sequence of Reals, and if $a \leq x_n \leq b \,\,\, \forall \,\,\, n\in\nn$,\\ Prove that $$a\leq\inflim x_n \leq b$$
        \sol As $x_n \geq a \,\,\,\forall n \in \nn$, Using lemma 1.b on $\{x_n\}$ and $\{a\}$ we get, \begin{align*}\inflim x_n \geq \inflim a = a \tag{1}\end{align*} Similarly,  as $b \geq x_n \,\,\,\forall n \in \nn$, Using lemma 1.b on $\{x_n\}$ and $\{b\}$ we get, \begin{align*}\inflim b = b \geq \inflim x_n \tag{2}\end{align*} From (1) and (2) have, 
        \begin{align*}
        a\leq\inflim x_n \leq b \tag*{\qed}
        \end{align*} 
        \end{enumerate} \clearpage
        \textbf{Problem 2.} Find the limits of the following sequences using Sandwich Theorem
        \begin{enumerate}[label={(\roman*)}]
        \item $$(2n)^\frac{1}{n}$$
        \sol Let
        \begin{align*}
            &x_n := (2n)^\frac{1}{n}\\
            &a_n := n^\frac{1}{n}\\
            &b_n := 2^\frac{1}{n}
        \end{align*}
        then, $x_n = a_n\cdot b_n$. Clearly, $a_n > 1 \,\,\,\forall \,\,\,n \in \nn$
        Also, 
        \begin{align*}
            \left(1 + \sqrt{\frac{2}{n}}\right)^n
            &> 1 + n\cdot \sqrt{\frac{2}{n}} + \frac{n(n-1)}{2} \cdot \frac{2}{n}\\
            &> 1 + n - 1\\
            & = n\\
            \therefore \left(1 + \sqrt{\frac{2}{n}}\right)^n > n \\ \Rightarrow 1 + \sqrt{\frac{2}{n}} > n^\frac{1}{n}
        \end{align*}
        Hence, 
        $$1 < n^\frac{1}{n} < 1 + \sqrt{\frac{2}{n}} \,\,\,\,\,\forall \,\, n \in \nn$$
        Also, $$\inflim 1 = \inflim  1 + \sqrt{\frac{2}{n}} = 1$$. Hence using sandwich theorem, 
        $$\inflim a_n=\inflim n^\frac{1}{n} = 1$$
        Also  trivially $1 < 2^\frac{1}{n}< n^\frac{1}{n}$. Thus, by sandwich theorem, $$\inflim b_n = \inflim 2^\frac{1}{n} = 1$$
        As $x_n = a_n \cdot b_n$, 
        $$\inflim x_n = \inflim a_n \cdot \inflim b_n = 1$$ or,
        $$\inflim (2n)^\frac{1}{n} = 1$$\qed \\
        \item $$\frac{n^2}{n!}$$ \\
        \sol Trivially $\frac{n^2}{n!} > 0$. Also, $n! < n^3 \,\,\, \forall n > 5$ we shall prove this by induction. \\
        Base Step :  When $n = 6$, $6! = 720 > 216 = 6^3$. i.e. the statement holds.\\
        Induction Step : Let's assume this holds for some $k \in \nn$ then, 
        \begin{align*}
            k! > k^3 &\Rightarrow (k+1)k! > (k+1)k^3\\
            &\Rightarrow (k+1)! > k^4 + k^3\\
            &\Rightarrow (k+1)! > k^3 + 3k^2 + 3k + 1 \tag{$k \geq 7 \Rightarrow k^4 > 7k^3 = 3k^3 + 3k^3 + k^3 > 3k^2 + 3k + 1$}\\
            &\Rightarrow (k+1)! > (k+1)^3
        \end{align*}
        thus, the statement also holds for k + 1. By invoking the principle of induction we can say that, 
        $n! > n^3 \,\,\, \forall \,\,\, n > 5$
        Thus, $\forall n > 5$
        $$0<\frac{n^2}{n!}<\frac{n^2}{n^3}=\frac{1}{n}$$
        Since,
        $$\inflim 0 = \inflim \frac{1}{n} = 0$$
        By sandwich theorem,
        \begin{align*}
            \inflim \frac{n^2}{n!} = 0 \tag*{\qed}
        \end{align*}
        \item $$\frac{2^n}{n!}$$
        \sol Trivially $\frac{2^n}{n!} > 0$. Also, $(n-1)! > 2^n\,\,\,\forall n > 5$ we shall prove this using induction \\ Base step : When $n = 6$, $(6 - 1)! = 5! = 120 > 64 = 2^6$\\
        Induction Step : Let's say this holds for some $k \in \nn$. then,
        \begin{align*}
            (k - 1)! > 2^k &\Rightarrow k(k-1)! > k\cdot2^k\\
            &\Rightarrow k! > 2\cdot2^k\\
            &\Rightarrow k! > 2^{k+1}\\
            &\Rightarrow ((k + 1) - 1)! > 2^{k + 1}
        \end{align*}
        hence, it holds for $(k+1)$ as well. Invoking the principle of mathematical induction we can say that this holds for all $n > 5$.\\
        Then, $\forall n >5$ we have, 
        $$0<\frac{2^n}{n!}<\frac{(n-1)!}{n!} = \frac{1}{n}$$
        Since, $$\inflim 0 = \inflim \frac{1}{n} = 0$$
        By sandwich theorem, $$\inflim \frac{2^n}{n!} = 0$$ \qed\\
        \item $$n^\frac{1}{n^2}$$ \\
        \sol Clearly, $$1 < n^\frac{1}{n^2}<n^\frac{1}{n}$$Since $$\inflim 1 = \inflim n^\frac{1}{n} = 1$$ by sandwich theorem, $$\inflim n^\frac{1}{n^2} = 1$$ \qed\\
        \item $$(n!)^\frac{1}{n^2}$$
        \sol Clearly, $$1<n!<n^n \Rightarrow 1 < (n!)^\frac{1}{n^2}<(n^n)^\frac{1}{n^2}=n^\frac{1}{n}$$ Since $$\inflim 1 = \inflim n^\frac{1}{n} = 1$$ by sandwich theorem, $$\inflim (n!)^\frac{1}{n^2} = 1$$ \qed
        \end{enumerate}
        \textbf{Problem 3.}
        \begin{enumerate}[label = {(\roman*)}]
        \item Using sandwich theorem prove that $$\lim_{n \to \infty} (2^n + 3^n)^{\frac{1}{n}} = 3$$
        \sol Clearly,
        $$3 = (0 + 3^n)^{\frac{1}{n}} < (2^n + 3^n)^\frac{1}{n}< (3^n + 
         3^n)^\frac{1}{n} = 3\cdot2^\frac{1}{n}$$
         Also, $$\lim_{n\to\infty} 3 = \lim_{n\to\infty} 3\cdot2^\frac{1}{n} = 3$$ Hence, by sandwich theorem,
         $$\lim_{n\to\infty} (2^n + 3^n)^\frac{1}{n} = 3$$ \qed\\
        \item Using sandwich theorem prove that $$\lim_{n\to\infty} \prod_{k=1}^{n} \frac{2k-1}{2k} = 0$$
        \sol Clearly,
        \begin{align*}
            \prod_{k=1}^{n} \frac{2k-1}{2k} > 0 \tag{1}
        \end{align*}
        Note that, 
        \begin{align*}
            ln\left(\prod_{k=1}^{n} \frac{2k-1}{2k}\right) 
            &= \sum_{k=1}^{n} ln \left( 1 - \frac{1}{2k}\right)\\
            &< \sum_{k = 1}^n -\frac{1}{2k} \tag{$ \because ln(1 + x) < x \,\,\, \forall x > -1$}\\
            &= -\frac{1}{2}\sum_{k=1}^n \frac{1}{k}
        \end{align*}
        Hence, 
        \begin{align*}
            ln\left(\prod_{k=1}^{n} \frac{2k-1}{2k}\right)  < -\frac{1}{2}\sum_{k=1}^n \frac{1}{k}\\
            \Rightarrow \prod_{k=1}^{n} \frac{2k-1}{2k}< e^{-\frac{1}{2}\sum_{k=1}^n \frac{1}{k}} \tag{2}
        \end{align*}
        From (1) and (2) we have,
        $$0<\prod_{k=1}^{n} \frac{2k-1}{2k}< e^{-\frac{1}{2}\sum_{k=1}^n \frac{1}{k}} \,\,\, \forall \,\,\,n \in \mathbb{N}$$
        Also, as $\lim_{n\to\infty} \sum_{k=1}^{n} \frac{1}{k} = \infty$, 
        $$\lim_{n\to\infty} e^{-\frac{1}{2}\sum_{k=1}^n \frac{1}{k}} = 0$$ 
        Thus, 
        $$\lim_{n\to\infty} 0 = \lim_{n\to\infty} e^{-\frac{1}{2}\sum_{k=1}^n \frac{1}{k}} = 0$$
        Hence by sandwich theorem, 
        $$\lim_{n\to\infty} \prod_{k=1}^{n} \frac{2k-1}{2k} = 0$$ \qed \\
        \end{enumerate} \clearpage
        \textbf{Problem 4.} Let $x_n$ be a sequence of positive real numbers such that $$\inflim \frac{x_{n+1}}{x_n} = x $$
        \begin{enumerate}[label = {(\alph*)}]
        \item If $0 \leq x < 1$, then prove that $\inflim x_n = 0$\\
        \sol Let $r \in \mathbb{R}$ such that $0<x<r<1$ then, for $\epsilon = r - x > 0$ $\exists k \in \nn$ such that $\forall n \geq k$ we have $\frac{x_{n + 1}}{x_n} < x + \epsilon = r$. Hence, $\forall n \geq k$, $$x_{n+1} < rx_n < x <r^2x_{n-1}<\cdot \cdot \cdot < r^{n - k +1}x_k$$ let us define $c :=  \frac{x_k}{r^k}$ then, $\forall k \in \nn$, $0< x_n < cr^n$ since $r<1$, $\inflim cr^n = \inflim 0 = 0$. Thus, by sandwich theorem, $$\inflim x_n = 0$$\qed\\
        \item If $x > 1$ prove that $\inflim x_n = \infty$\\
        \sol Let $r \in \mathbb{R}$ such that $x>r>1$ then, for $\epsilon = x - r > 0$ $\exists k \in \nn$ such that $\forall n \geq k$ we have $\frac{x_{n + 1}}{x_n} > x - \epsilon = r$. Hence, $\forall n \geq k$, $$x_{n+1} > rx_n > x >r^2x_{n-1}>\cdot \cdot \cdot > r^{n - k +1}x_k$$ let us define $c :=  \frac{x_k}{r^k}$ then, $\forall k \in \nn$, $x_n > cr^n$ since $r>1$, $\inflim cr^n = \infty$. Thus, $$\inflim x_n = \infty$$\qed\\
        \end{enumerate}
        \textbf{Problem 5.} \\
        (a) Show that the following sequences diverge to $\infty$\\
        \begin{enumerate}[label={(\roman*)}]
          \item $$2^n$$\\
        \sol Let $M > 0$ be given. Let $N \in \mathbb{N}$ such that $N > \log_2M$ (Archimidean Principle). Then, $\forall   n\geq N$, $2^n \geq 2^N > 2^{\log_2M} = M$. Hence, $$\inflim 2^n = \infty$$\qed\\
        \item $$a^n$$\\
        \sol Let $M > 0$ be given. Let $N \in \mathbb{N}$ such that $N > \log_aM$ (Archimidean Principle). Then, $\forall   n\geq N$, $a^n \geq a^N > a^{\log_aM} = M$. Hence, $$\inflim a^n = \infty$$\qed\\
        \item $$(n!)^\frac{1}{n}$$\\
        \sol We claim that $n! > n^\frac{n}{2} \forall n > 2$. We will prove this using induction. Clearly, $3! = 6 > \sqrt{27} = 3^\frac{3}{2}$ thus, the statement holds for $n = 3$. Let's say it's true for $n = k$ then,
        \begin{align*}
            k! > k^\frac{k}{2} &\Rightarrow (k!)^2 > k^k \\
            &\Rightarrow (k+1)^2\cdot(k!)^2 > (k+1)^2\cdot k^k\\
            &\Rightarrow ((k+1)!)^2 > (k+1)^2\cdot k^k \tag{1}
        \end{align*}
        Note that $\forall k > 2$, 
        \begin{align*}
            (k + 1) > 3 >\left(1+\frac{1}{k}\right)^k &\Rightarrow (k + 1) > \frac{(k+1)^k}{k^k}\\
            &\Rightarrow k^k > (k + 1)^{(k-1)}\\
            &\Rightarrow (k + 1)^2 \cdot k^k > (k + 1)^{(k+1)} \tag {2}
        \end{align*}
        Using (2) in (1) we have,
        $$((k+1)!)^2 > (k+1)^{(k+1)} \Rightarrow (k+1)! > (k+1)^\frac{(k+1)}{2}$$
        thus, the statement holds true for $k+1$. Hence, invoking the principle of induction we have $$n! > n^\frac{n}{2} \Rightarrow (n!)^\frac{1}{n} > \sqrt{n}$$ 
        Since, $\inflim \sqrt{n} = \infty$ , $$\inflim (n!)^\frac{1}{n} = \infty$$\qed\\
        \end{enumerate}
        (b) Show that the sequence $x_n = -n^2$ diverges to $-\infty$\\
        \sol Let $M < 0$ be given. Let $N \in \nn$ such that $N > \sqrt{-M}$ (Archimedean property). Then, $\forall n \geq N$ we have,
        \begin{align*}
            n^2 \geq N^2 &\Rightarrow n^2 > -M\\
            &\Rightarrow -n^2 < M\\
            &\Rightarrow x_n < M
        \end{align*}
        Hence, 
        $$\inflim x_n = -\infty$$\qed\\
        \begin{center}
            \Large{\textbf{A useful result to show that a sequence diverges to $\infty$}}
        \end{center}
        \begin{lemma}
             Let $x_n$ and $y_n$ be two real sequences such that, $x_n \geq y_n$ $\forall n \geq m$ for some $m \in \nn$. Then, $$\inflim y_n = \infty \Rightarrow \inflim x_n = \infty$$
        \end{lemma}
        \begin{proof}
            Let $M > 0$ be given. Since $\lim y_n = \infty$, $\exists N_y \in \nn$ such that $y_n > M$ $\forall n \geq N_y$. Let us choose $N \in \nn$ as $N > max\{N_y, m\}$. Then, $\forall n \geq N$ we have, $x_n \geq y_n>M \Rightarrow x_n > M$. Hence $\lim x_n = \infty$. This proves our claim.
        \end{proof}
        This result has been used in the solutions of 4.(b) and 5.(a)(iii) \\ \\ 
        Note that a  similar result can be obtained that states, if $x_n \leq y_n$, $\forall n \geq m$, $\lim y_n = -\infty \Rightarrow \lim x_n = -\infty$.
        \begin{proof}
            Easy
        \end{proof}
            
\end{document}