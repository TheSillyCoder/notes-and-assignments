\documentclass[10pt]{article}

\usepackage[T1]{fontenc}
\usepackage{geometry}
\usepackage{graphicx}
\usepackage{amssymb}
\usepackage{amsmath}
% \usepackage{enumitem}
\usepackage[english]{babel}
\usepackage{amsthm}
\usepackage[hidelinks]{hyperref}
% \usepackage{graphicx}


\newtheorem*{lemma}{Lemma}
\title{MA1202 Notes - Ordinary Differential Equations}
\author{Debayan Sarkar}
\date{May 5, 2023}

\geometry{a4paper, margin=0.8in}
\setlength{\parindent}{0pt}
\begin{document}
    \maketitle
    \tableofcontents
    \clearpage
    \section{First Order ODEs}
        \subsection{Linear First Oder ODEs}
            \subsubsection{Constant Coefficients}
                This equations are of the form: $$y' = ay + b$$ In this case, we use an integrating factor $\mu(x) = e^{-ax}$  such that, after multiplying by $\mu(x)$ on both sides, the equation becomes
                \begin{align*}
                    &e^{-ax}(y' - ay) = be^{-ax}\\
                    \Rightarrow &(ye^{-ax})' = -(\frac{be^{-ax}}{a})'\\
                    \Rightarrow &ye^{-ax} + \frac{be^{-ax}}{a} = C\\
                \end{align*}
                Hence, the solution to the given differential equation is, $$y(x) = Ce^{ax} - \frac{b}{a}$$
        \subsubsection{ Variable Coefficients}
            These equations are of the form $$y' = a(x)y + b(x)$$.
            In this case as well, we use an integrating factor $\mu(x)$
            To obtain this integrating factor we must understand the motivation behind it
            We are looking for a $\mu(x)$ such that, 
            $$\mu(y' - a(x)y) = (\mu y)'$$
            on simplifying we get, 
            \begin{align*}
                \mu y' - a(x)y\mu = \mu y' + \mu' y
                \Rightarrow \frac{\mu'}{\mu} = -a(x)y
                \Rightarrow (ln(\mu))' = -a(x)y
            \end{align*}
            Let's say we are solving an initial value problem where $y(x_0) = y_0$
            Then, 
            $$\mu(x) = e^{-A(x)}$$
            Where 
            $$A(x) = \int_{x_0}^{x}a(x)dx$$
            Using this integrating factor, the solution of the initial value problem is,
            $$y(x) = e^{A(x)}[y_0 + \int_{x_0}^{x}e^{-A(x)}b(x)dx ]$$
        \subsection{Exact Differential Equations}
            \subsubsection{Definition}
                The equation 
                $$N(t, y(t))y' + M(t, y(t)) = 0$$
                is said to be exact if and only if
                $$\partial_t N = \partial_yM$$
            \subsubsection{Poincarre Lemma }
                The Poincarre lemma states that, there exists a potential function $\psi(t,y(t)) = C$ such that,
                $$\partial_y\psi = N$$$$\partial_t\psi = M$$
                
                Hence, $\psi(t, y(t)) = C$ is the solution of our exact differential equations
                
                \textbf{Example :}\\
                Find all the solutions to the equation 
                $$[sin(t) + t^2 e^{y(t)} - 1]y'(t) + y(t)cos(t) + 2te^{y(t)}$$
                \textbf{Solution : }
                As we can see here, 
                $$ N = sin(t) + t^2 e^{y(t)} - 1$$
                and,
                $$M = y(t)cos(t) + 2te^{y(t)}$$
                First we check if the given equation is exact.
                $$\partial_tN = cos(t) + 2te^{y(t)}$$
                $$\partial_yM = cos(t) + 2te^{y(t)}$$
                Since $\partial_tN = \partial_yM$, the given equation is exact. Therefore, by Poincarre lemma a function $\psi(t, y(t)) = C_1$ exists such that, $\partial_y\psi=N$ and $\partial_t\psi=M$
                Hence, $$\partial_y\psi = sin(t) + t^2e^{y(t)} - 1$$
                From this if we integrate both sides w.r.t. y, we get,
                $$\psi(t, y(t)) = y(t)sin(t) + t^2e^{y(t)} - y(t) + F(t)$$
                also,
                $$\partial_t\psi = y(t)cos(t) + 2te^{y(t)} + F'(t) = M = y(t)cos(t) + 2te^{y(t)}$$
                therefore, $F'(x) = 0$ or $F(x) = C_2$
                Hence, 
                $$\psi(t, y(t)) = y(t)sin(t) + t^2e^{y(t)} - y(t) + C_2 = C_1$$
                or,
                $$y(t)sin(t) + t^2e^{y(t)} - y(t) = C$$
                is the solution of the given differential equation.
        \subsection{Method of Integrating Factor}
            Let us consider a first order ODE $$N(t, y(t))y'(t) + M(t, y(t)) = 0$$ which is not exact
            then, we try to find an integrating factor such that, multiplying the equation by the integrating factor makes it exact. Let this integrating factor be $\mu(t)$. Multiplying we get,
            $$\mu(t)N(t, y(t))y'(t) + \mu(t)M(t, y(t)) = 0$$
            Let 
            $$\tilde{N} = \mu(t)N$$ and,
            $$\tilde{M} = \mu(t)M$$
            Then, for this new equation to be exact, $\partial_t\tilde{N} = \partial_y\tilde{M}$ which on simplifying yields,
            $$\frac{\mu'}{\mu} = \frac{\partial_yM-\partial_tN}{N}$$
            If this is a function of $t$ only, then it can be solved just like an exact differential equation.
    \section{Second Order ODEs}
        Given functions $p,q,f : (t_1, t_2) \to \mathbb{R}$ a second order ordinary differential equation in the unknown variable $y : (t_1, t_2) \to \mathbb{R}$ is given by $$y'' + p(t)y' + q(t)y = f(t)$$ Here $p(t)$ and $q(t)$ can be constant or variable. If $f(t) = 0$, it's called a homogeneous equation.
        \subsection{Linear Dependancy of Solutions}
            If the functions $y_1$ and $y_2$ are two solutions of the equation 
            $$y'' + a(t)y' + b(t) = 0$$
            then $c_1y_1(t) + c_2y_2(t)$ is also a solution $\forall c_1, c_2 \in \mathbb{R}$.  
            \subsubsection{The Wronskian}
            The Wronskian of two functions $y_1$ and $y_2$ is defined as the determinant,
            $$
            W_{y_1y_2} = 
            \begin{pmatrix}
            y_1  & y_2 \\
            y_1' & y_2'
            \end{pmatrix}
            $$
            If $W_{y_1y_2}=0$, then $y_1$ and $y_2$ are said to be linearly dependant.
        \subsection{Homogeneous Equations}
            \subsubsection{Constant Coefficients}
                The equation will be of the form $$y'' + ay' + by = 0$$
                Guessing $y = e^{rt}$ as a solution we obtain the \textbf{characteristic polynomial} of the equation to be 
                $$P(r) = r^2 + ar +b$$
                The \textbf{characteristic equation} is $$P(r) = 0$$
                let the solutions to $P(r) = 0$ be $r_+$ and $r_-$. then,
                $$r_{\pm} = \frac{1}{2}(-a \pm \sqrt{a^2-4b})$$
                \textbf{Case 1 :} $r_{+} = r_{-}$
                $$y(t) = (c_0 + c_1 t)e^{r_+t}$$
                \textbf{Case 2 :} $r_+ \ne r_-$
                $$y(t) = c_0e^{r_+t} + c_1e^{r_-t}$$
                \begin{center}
                    \Large{\textbf{Variable Coefficients}}
                \end{center}
            \subsubsection{Reduction of Order Method}
                If $y_1(t)$ is a solution of $y'' + a(t)y' + b(t)y = 0$ then, we can assume the other solution  $y_2(t)$ as  $$y_2(t) = v(t)y_1(t)$$
                Substituting this value into the original equation and simplifying leads to a first order equation in $w$ where $w = v'$ . 
                
                \textbf{Example}\\
                Find the other linearly independent solution of the equation 
                $$t^2y'' + 2ty' - 2y = 0$$
                if one of the solutions is given to be $y_1(t) = t$.
                
                \textbf{Solution :}
                Let  $y_2 = vt$
                then, $y_2' = v + v't$ and $y_2'' = 2v' + v''t$
                By substituting these values in the original equation we get,
                \begin{align*}
                2t^2v' + v''t^3 + 2tv + 2v't^2 - 2vt = 0 
                \Rightarrow &v''t^3 + 4t^2v'= 0\\
                \Rightarrow &\frac{v''}{v'} = -\frac{4}{t}\\
                \Rightarrow &v' = ct^{-4}\\
                \Rightarrow &v = c_0t^{-3} + c_1
                \end{align*}
                $$\therefore y = \frac{c_0}{t^2} + c_1t$$
                hence, the other fundamental solution is $$y_2 = \frac{1}{t^2}$$
        \subsection{Non-homogeneous Equations}
            These equations are of the form $$y'' + p(t)y' + q(t)y = f(t)$$
            Let $\mathcal{L}$ be an operator such that $$\mathcal{L}(y) = y'' + + p(t)y' + q(t)y$$ Then, for homogeneous equations $\mathcal{L}(y) = 0$ and for non-homogeneous equations $\mathcal{L}(y) = f(t)$. This $\mathcal{L}$ operator is linear. Hence, if $y_1$ and $y_2$ are solutions of the homogeneous equation, then $\mathcal{L}(c_1y_1 + c_2y_2) = c_1\mathcal{L}(y_1) +c_2\mathcal{L}(y_2)$. If $y_p$ is the solution of the equation $\mathcal{L}(y) = f(t)$, then the general solution of this equation is given by $$y = c_1y_1 + c_2y_2 + y_p$$ 
            If the function $f$ (also known as the source function) can be written as $f(t) = f_1(t) + \cdot \cdot \cdot + f_n(t)$, and if there exist $y_i$ such that $\mathcal{L}(y_i) = f_i(t)$, then $y_p = y_1 + \cdot \cdot \cdot + y_n$ satisfies equation $\mathcal{L}(y_p) = f(t)$
            \subsubsection{Method of Unknown Coefficients}
                To Solve a non-homogeneous second order ODE using the method of Unknown Coefficients, we follow the following steps :
                \begin{enumerate}
                    \item Find the general solution $y_h$ of the homogeneous equation given by $\mathcal{L}(y_h) = 0$
                    \item If the source function can be decomposed into multiple functions like $f = f_1 + \cdot \cdot \cdot + f_n$ where $n \geq 1$, then find $y_{p_i}$ as  solution of the equation $\mathcal{L}(y_{p_i}) = f_i(t)$ for all $1\leq i\leq n$.
                    \item Given the source function $f_i$, guess the solution $y_{p_i}$ based on the table given below.
                    \begin{center}
                        \begin{tabular}{| c | c |}
                        \hline
                             $f_i(t)$ & $y_{p_i}$ \\
                             \hline \hline
                             $Ke^{at}$ & $ke^{at}$ \\
                             \hline
                             $Kt^m$ & $k_mt^m + k_{m-1}t^{m-1} + \cdot \cdot \cdot + k_0$\\
                             \hline
                             $Kcos(bt)$ or $Ksin(bt)$ & $k_1 sin (bt) + k_2 cos(bt)$\\
                             \hline
                             $Ke^{at} sin (bt)$ or $Ke^{at} cos(bt)$ & $e^{at}(k_1 sin (bt) + k_2 cos(bt))$ \\
                             \hline 
                             $Kt^me^{at}$ & $e^{at}(k_mt^m + k_{m-1}t^{m-1} + \cdot \cdot \cdot + k_0)$\\
                             \hline
                             $Kt^m cos(bt)$ or $Kt^m sin(bt)$ & $(k_mt^m + \cdot \cdot \cdot + k_0)(a_1 sin (bt) + a_2 cos(bt))$\\
                             \hline
                        \end{tabular}
                    \end{center}
                    \item If a guess $y_{p_i}$ happens to be a solution of $\mathcal{L}(y_{p_i}) = 0$, try $t^k y_{p_i}$ with $k \in \mathbb{N}$ sufficiently large so that $\mathcal{L}(y_{p_i}) \neq 0$.
                    \item Impose the equation $\mathcal{L}(y_{p_i}) = f_i(t)$ to find out the unknown coefficients $k_1, \cdot \cdot \cdot , k_n$ and then compute $y_p = y_1 + \cdot \cdot \cdot + y_n$.
                    \item The solution to the original problem is then given by $y = y_h + y_p$.
                \end{enumerate}
                \textbf{Example}\\
                Solve: $$y'' - 3y' - 4y = 3e^{4t}$$
                \textbf{Solution: }First we obtain the solution of the homogeneous equation as $y_h = c_1e^{4t} + c_2e^{-t}$ \\
                The source function $f(t) = 3e^{4t}$ is not decomposable as the sum of simpler functions. So, from the table given above we guess the solution $y_p = ke^{4t}$. But this is already a solution of the homogeneous equation. So we modify our guess to $y_p = kte^{4t}$. Now, we have to determine the value of $k$. $y_p' = k(e^{4t} + 4te^{4t})$ and $y_p'' = k(8e^{4t} + 16te^{4t})$. Substituting these values in the given equation we get,
                \begin{align*}
                    &k(8e^{4t}+ 16te^{4t}) - 3k(e^{4t} + 4te^{4t}) - 4kte^{4t} = 3e^{4t} \\
                    \Rightarrow &8k + 16kt - 3k - 12kt - 4kt = 3 \\
                    \Rightarrow &5k = 3\\
                    \Rightarrow & k = \frac{3}{5}
                \end{align*}
                $$\therefore y_p = \frac{3}{5}te^{4t}$$
                And hence, the general solution of the equation can be written as $$y = c_1e^{4t} + c_2e^{-t} + \frac{3}{5}te^{4t}$$
            \subsubsection{Method of Variation of parameters}
                Let $y_1$ and $y_2$ be the fundamental solutions of the homogeneous equation $\mathcal{L}(y) = 0$. Then $y_p = u_1y_1 + u_2y_2$ is a particular solution of the non-homogeneous equation $\mathcal{L}(y) = f(t)$ where $u_1$ and $u_2$ are defined as,
                $$u_1 = -\int\frac{y_2f(t)}{W_{y_1y_2}}dt$$
                $$u_2 = \int\frac{y_1f(t)}{W_{y_1y_2}}dt$$
                \textbf{Example}\\
                $$y'' - 5y' + 6y = 2e^t$$
                \textbf{Solution: } The solutions to the homogeneous equations are, $y_1 = e^{3t}$ and $y_2 = e^{2t}$ \\
                Also,
                $$
                W_{y_1y_2} = 
                \begin{pmatrix}
                y_1  & y_2 \\
                y_1' & y_2'
                \end{pmatrix}
                =
                \begin{pmatrix}
                e^{3t}  & e^{2t} \\
                3e^{3t} & 2e^{2t}
                \end{pmatrix}
                = 2e^{5t} - 3e^{5t} = -e^{5t}
                $$
                Hence,
                $$u_1 = -\int \frac{e^{2t} \cdot 2e^{t}}{-e^{5t}} dt = 2\int e^{-2t}dt = -e^{-2t}$$
                $$u_2 = \int \frac{e^{3t} \cdot 2e^{t}}{-e^{5t}} dt = -2\int e^{-t}dt = 2e^{-t}$$
                Therefore, the particular solution is $y_p = (-e^{-2t})(e^{3t})+(2e^{-t})(e^{2t}) = e^t$
                and the general solution of the equation is, $$y = c_1e^{3t} + c_2e^{2t} + e^t$$
    \section{Power Series Solutions}
        Let the $P, Q, R : (x_1, x_2) \to \mathbb{R}$ be functions and a second order differential equation in an unknown variable $y: (x_1, x_2) \to \mathbb{R}$ be defined as $$P(x)y'' + Q(x)y' + R(x)y = 0$$
        Let $x_0 \in (x_1, x_2)$. If $P(x_0) = 0$, we say that $x_0$ is a \textbf{singular point}. If $P(x_0) \neq 0$, we say that $x_0$ is a \textbf{regular point}. If we have $x_0$ as a regular point, we can substitute $y(x) = \sum_{n=0}^\infty a_n(x-x_0)^n$ as a solution, and compute the power series coefficients, to arrive at a solution of the differential equation.
        \subsection{Examples of power series solutions}
        Let's consider a few examples. \\
        \textbf{Example 1.} \\
        $$y'' + y = 0$$
        \textbf{Solution: }For this equation, $P(x) = 1 \forall x \in \mathbb{R}$. So, for convenience we will choose $x_0 = 0$ for the solution. So, let $$y = \sum_{n=0}^\infty a_nx^n$$ then, 
        $$y'' = \sum_{n=2}^\infty n(n-1)a_nx^{n-2} = \sum_{n=0}^\infty (n+2)(n+1)a_{n+2}x^n$$
        Substituting the values of $y$ and $y''$ in $y'' + y = 0$ we get,
        \begin{align*}
            &\sum_{n=0}^\infty (n+2)(n+1)a_{n+2}x^n + \sum_{n=0}^\infty a_nx^n = 0 \\
            \Rightarrow &\sum_{n=0}^\infty [(n+2)(n+1)a_{n+2} + a_n]x^n = 0\\
            \Rightarrow &(n+1)(n+1)a_{n+2} + a_n = 0
        \end{align*}
        For $n = 2k$,
        \begin{align*}
            &(2k+2)(2k+1)a_{2k+2} + a_{2k} = 0 \\
            \Rightarrow &a_{2k+2} = -\frac{a_{2k}}{(2k+2)(2k+1)} \\
            \Rightarrow &a_{2k} = (-1)^k\frac{a_0}{(2k)!}
        \end{align*}
        A similar result can be obtained for $n = 2k + 1$ 
        $$a_{2k + 1} = (-1)^k \frac{a_1}{(2k+1)!}$$
        Then, the obtained solution can be written as, 
        $$y = a_0\left(1 - \frac{x^2}{2!} + \frac{x^4}{4!} + \cdot \cdot \cdot \right) + a_1\left(x - \frac{x^3}{3!} + \frac{x^5}{5!} \cdot \cdot \cdot \right) = a_0\cos x + a_1 \sin x$$

        \textbf{Example 2.} \\
        $$(x^2 + 1)y'' - 4xy' + 6y = 0$$
        \textbf{Solution: }Again, like the previous example $P(x) = x^2 + 1 \neq 0 \,\, \forall x \in \mathbb{R}$. So, again we will choose to expand around $x_0 = 0$ for convenience. In this example we will compute each term in the equation as that is an easier way (for me) of dealing with the summations in power series 
        $$6y = \sum_{n=0}^\infty 6a_n x^n $$
    	$$4xy' = \sum_{n=1}^\infty 4x\cdot na_nx^{n-1} = \sum_{n=0}^\infty 4na_nx^n$$
    	$$(x^2 + 1)y'' = (x^2 + 1)\sum_{n=2}^\infty n(n-1)a_nx^{n-2} = \sum_{n=0}^\infty n(n-1)a_nx^n + \sum_{n=0}^\infty (n+2)(n+1)a_{n+2}x^n$$
	    Substituting these values in $(x^2 + 1)y'' - 4xy' + 6y = 0$ we get, 
    	\begin{align*}
    		&\sum_{n=0}^\infty [n(n-1)a_n + (n+2)(n+1)a_{n+2} - 4na_n + 6a_n = 0 ]x^n = 0 \\
    		\Rightarrow &(n(n-1)-4n+6)a_n + (n+2)(n+1)a_{n+2} = 0 \\
    		\Rightarrow &(n-2)(n-3)a_n + (n+2)(n+1)a_{n+2} = 0
        \end{align*}
        For $n=0$,
        $$6a_0 + 2a_2 = 0 \Rightarrow a_2 = -3a_0$$
        For $n = 1$
        $$2a_1 + 6a_3 = 0 \Rightarrow a_3 = -\frac{a_1}{3}$$
        For $n = 2$
        $$0\cdot a_2 + 12a_4 = 0 \Rightarrow a_4 = 0 \Rightarrow a_{2k} = 0 \,\,\, \forall k \geq 2$$
        For $n = 3$,
        $$0\cdot a_3 + 20a_5 = 0 \Rightarrow a_5 = 0 \Rightarrow a_{2k+1} = 0 \,\,\, \forall k \geq 2$$
        Hence the obtained solution is, 
        $$y = a_0 + a_1x -3a_0x^2 -\frac{a_1}{3}x^3 = a_0(1 - 3x^2) + a_1 \left(x - \frac{x^3}{3}\right)$$
        \subsection{Cauchy-Euler Equidimensional Equations}
            These differential equations have the form $$(x-x_0)^2y'' + p_0(x-x_0)y' + q_0y = 0$$ Note that $x_0$ is not a regular point, but a singular point. To find the solutions to this equation, substitute $y=(x-x_0)^r$ into the equation. After simplifying we get the Euler characteristic polynomial $P(r)$,
            $$P(r) = r(r-1) + p_0r  + q_0 = 0$$
            Let the solutions to this quadratic equation in $r$ be $r_+$ and $r_-$. Then there can be three cases for which the corresponding solutions of the differential equation have been listed below :\\
            \textbf{Case 1} They are real and distinct\\
            $$y = c_0(x-x_0)^{r_+} + c_1(x-x_0)^{r_-}$$
            \textbf{Case 2} They are real and identical \\
            Let $r_+ = r_- = r$ then,
            $$y = (x-x_0)^{r}(c_0 + c_1 \ln (x-x_0))$$
            \textbf{Case 3} They are complex conjugates \\
            Let $r_\pm = \alpha \pm i \beta$ then,
            $$y = x^\alpha(c_0\cos (\beta\ln(x-x_0)) + c_1\sin (\beta \ln (x-x_0)))$$
            \textbf{Example} \\
            $$2x^2y'' + 3xy' - 15y = 0$$
            \textbf{Solution: } On substituting $y = x^n$ in the equation we get,
            \begin{align*}
                &2n(n-1) + 3n - 15 = 0 \\
                \Rightarrow &2n^2 - 2n + 3n - 15 = 0 \\
                \Rightarrow &2n^2 + n - 15 = 0\\
                \Rightarrow &(2n - 5)(n + 3) = 0 \\
                \Rightarrow & n = \frac{5}{2}, -3
            \end{align*}
            Hence, the solution to the equation is, 
            $$y = c_0x^\frac{5}{2} + c_1x^{-3}$$
        \subsection{Equations with Regular-Singular points}
        % \subsubsection{Definition} 
        Let $x_0$ be a singular point of the equation $$P(x)y'' + Q(x)y' + R(x)y = 0$$ and let
        \begin{align*}
            f(x) = \frac{(x-x_0)Q(x)}{P(x)}\\
            g(x) = \frac{(x-x_0)^2R(x)}{P(x)}
        \end{align*}
        then, $x_0$ is called \textbf{regular-singular} point if and only if the limits $\lim_{x\to x_0} f(x)$ and $\lim_{x\to x_0} g(x)$ are finite, and both f(x) and g(x) admit convergent Taylor series expansions around $x_0$.\\ \\
        
        \textbf{Remark.} Every equation $P(x)y'' + Q(x)y' + R(x)y = 0$ with a \textbf{regular-singular} point at $x_0$ is close to an Euler equation.
        \begin{proof}
            For $x \neq x_0$ dividing both sides by $P(x)$ we get, $y'' + \frac{Q(x)}{P(x)}y' + \frac{R(x)}{P(x)}y = 0$ multiplying it by $(x-x_0)^2$ we get, 
            $$(x-x_0)^2 + (x-x_0) \left[\frac{(x-x_0)Q(x)}{P(x)}\right]y' + \left[\frac{(x-x_0)^2R(x)}{P(x)}\right]y = 0$$
            Since $x_0$ is a regular-singular point, as $x\to x_0$, the factors $\frac{(x-x_0)Q(x)}{P(x)} \to p_0 (say)$ and $\frac{(x-x_0)^2R(x)}{P(x)} \to q_0 (say)$. Then, the equation becomes 
            $$(x-x_0)^2y'' + p_0(x-x_0)y' + q_0y = 0$$
        \end{proof}
        \textbf{Example}\\
        Find the regular-singular points of the differential equation $$(x+2)^2(x-1)y'' + 3(x-1)y' + 2y = 0$$
        \textbf{Solution: } We have two singular points, $x_0 = -2$ and $x_1 = 1$. \\
        When $x_0 = -2$, 
        $$\lim_{x\to-2} \frac{3(x+2)(x-1)}{(x+2)^2(x-1)} = \lim_{x\to-2} \frac{3}{x+2} = \pm \infty$$
        Hence, $x_0 = -2$ is not a regular-singular point.\\
        When $x_1 = 1$,
        $$\lim_{x\to1} \frac{3(x-1)(x-1)}{(x+2)^2(x-1)} = \lim_{x\to1} \frac{3(x-1)}{x+2} = 0$$ 
        $$\lim_{x\to1} \frac{2(x-1)^2}{(x+2)^2(x-1)} = \lim_{x\to1} \frac{2(x-1)}{x+2} = 0$$ 
        and both $\frac{3(x-1)}{x+2}$ and $\frac{2(x-1)}{x+2}$ have covergent Taylor series around $x_1 = 1$. Hence, $x_1 = 1$ is a regular singular point of the equation.
\end{document}
